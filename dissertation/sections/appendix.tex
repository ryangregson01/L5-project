\section{Appendix}

\begin{table*}[]
\centering
\caption{Mistral and Llama 2 have been queried and demonstrate their understanding of sensitive personal information.}
\label{table:model_justify}
\begin{tabular}{|p{6.2in}|}
\hline
\textbf{Prompt}                                                       \\
{[}INST{]} What does sensitive personal information mean? {[}/INST{]} \\ \hline
\textbf{Mistral answer}                                               \\
\begin{tabular}[c]{@{}p{6.2in}@{}}Sensitive personal information refers to data that is highly personal or sensitive in nature, and is typically protected by laws and regulations that govern the collection, use, and disclosure of personal data. Here are some examples of sensitive personal information:\\ 1. Financial information: This includes credit card numbers, bank account numbers, and other financial information that can be used to access an individual's financial accounts.\\ 2. Health information: This includes medical records, health insurance information, and other personal health data that is protected by the Health Insurance Portability and Accountability Act (HIPAA) in the United States.\\ 3. Biometric information: This includes biometric data such as fingerprints, facial recognition data, and other unique identifiers that can be used to authenticate an individual's identity.\\ 4. Personal identity documents: This includes passports, driver's licenses, and other government-issued identification documents that can be used to verify an individual's identity.\\ 5. Location data: This includes location information from GPS-enabled devices, mobile phones, and other sources that can be used to track an individual's location and movements.\\ 6. Children's information: This includes personal information about children, such as their names, addresses, and birthdates, that is protected by the Children's Online Privacy Protection Act (COPPA) in the United States.\\ 7. Sexual orientation or gender identity: This includes personal information about an individual's sexual orientation or gender identity, which is protected by various laws and regulations around the world.\\ 8. Racial or ethnic information: This includes personal information about an individual's race or ethnicity, which is protected by various laws and regulations around the world.\\ 9. Political beliefs or activities: This includes personal information about an individual's political beliefs or activities.\end{tabular} \\ \hline
\textbf{Llama 2 answer}                                               \\
\begin{tabular}[c]{@{}p{6.2in}@{}}Sensitive personal information refers to data that is highly confidential and personal to an individual. This type of information is considered sensitive because its disclosure could result in harm, embarrassment, or invasion of privacy. Some common examples of sensitive personal information include:\\ 1. Social Security numbers\\ 2. Financial account information, such as credit card or bank account numbers\\ 3. Health information, including medical records and conditions\\ 4. Biometric data, such as fingerprints or facial recognition information\\ 5. Passwords or other authentication information\\ 6. Personal identification numbers (PINs)\\ 7. Home addresses or other contact information\\ 8. Racial or ethnic origin\\ 9. Religious or philosophical beliefs\\ 10. Sexual orientation or gender identity\\ \\ It's important to protect sensitive personal information from unauthorized access, use, or disclosure.\end{tabular} \\ \hline
\end{tabular}
\end{table*}




\begin{table*}[]
\centering
\caption{Mistral and Llama 2 have been queried and demonstrate their understanding of FOIA Section 40 as a reason for needing to protect sensitive personal information.}
\label{table:model_justify2}
\begin{tabular}{|p{6.2in}|}
\hline
\textbf{Prompt}                                     \\
{[}INST{]} Describe UK FOIA Section 40. {[}/INST{]} \\ \hline
\textbf{Mistral answer}                             \\
\begin{tabular}[c]{@{}p{6.2in}@{}}The United Kingdom's Freedom of Information Act (FOIA) contains several exemptions that allow public authorities to withhold certain information from disclosure. One such exemption is Section 40, which relates to personal information.\\ \\ Section 40(1) of the FOIA provides that information is exempt if its disclosure would breach one of the data protection principles, as set out in the Data Protection Act 1998 (DPA). These principles include requirements that personal data be obtained and processed fairly and lawfully, be accurate and kept up to date, and not be disclosed without the consent of the data subject, unless one of the specified conditions is met.\\ \\ Section 40(2) of the FOIA provides an additional exemption for personal data that is held by public authorities for the purposes of preventing or detecting crime. In this case, disclosure of the information would be exempt if it would be likely to prejudice the prevention or detection of crime, or the apprehension or prosecution of offenders.\\ \\ Section 40(3) of the FOIA provides an exemption for personal data that is held by public authorities for the purposes of safeguarding the physical or mental health of an individual. In this case, disclosure of the information would be exempt if it would be likely to cause serious harm to the physical or mental health of the individual, or to someone else, if disclosed.\\ \\ Section 40(4) of the FOIA provides an exemption for personal data that is held by public authorities for the purposes of assessing or collecting tax or duty. In this case, disclosure of the information would be exempt if it would be likely to prejudice the assessment or collection of tax or duty.\\ \\ It is important to note that the exemptions in Section 40 are not absolute.\end{tabular} \\ \hline
\textbf{Llama 2 answer}                             \\
\begin{tabular}[c]{@{}p{6.2in}@{}}Section 40 of the Freedom of Information Act (FOIA) in the United Kingdom provides an exemption for certain types of information that are considered to be sensitive or confidential. This exemption is known as "section 40(2)" or "section 40(3)" depending on the specific circumstances\\ \\ Section 40(2) applies when the information requested is held by a public authority, but is not available for release because it is protected by one of the following exemptions:\\
\begin{itemize}
    \item Exemption 1: Information that is protected by legal professional privilege (LPP) 
    \item Exemption 2: Information that is confidential or privileged in the context of legal proceedings
    \item Exemption 3: Information that is confidential or privileged in the context of a contract or other agreement
    \item Exemption 4: Information that is confidential or sensitive in the context of a trade secret or commercial confidentiality
\end{itemize}
\\ Section 40(3) applies when the information requested is held by a public authority, but is not available for release because it is protected by one of the following exemptions:\\
\begin{itemize}
    \item Exemption 5: Information that is personal data and is protected under the Data Protection Act 2018
    \item Exemption 6: Information that is sensitive personal information and is protected under the Data Protection Act 2018
    \item Exemption 7: Information that is protected by the right to privacy and confidentiality under the Human Rights Act 1998
\end{itemize}
\\ In both cases, the public authority is not required to confirm or deny the existence of the information, but must instead issue a notice stating that the information is exempt from disclosure under section 40. The notice must also explain why the information is exempt and provide details of any relevant advice or guidance that is available.\end{tabular} \\ \hline
\end{tabular}
\end{table*}

